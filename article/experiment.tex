\section{Experiment}
\subsection{Equal data} \label{equal-data}
When measuring the performance of the different methods it is essential that the datasets are the same.
Without them being equal the results of the experiments wont be worth comparing.
Both PUP and LightGCN utilized the yelp dataset according to their papers.
We did not get the dataset from the PUP implementation and were unable to find the original yelp-2018 dataset from the LightGCN implementation were the price and category was still attached.
We therefore decided to utilize the yelp-2020 dataset, in which users rank businesses between 1 to 5 stars.
Inspired by the PUP method we also decided to only use businesses with the restaurant category.
What is left is a set of businesses with one or more subcategories that relate to what type of restaurant they are.
Like PUP we cut off any extra subcategories so that each business now only has one subcategory.
The single subcategory that an item possesses is referred to as its category in the rest of this paper.
\\
PUP has a 60\% training-, 20\% validation- and 20\% test-data split while LightGCN has a 70\% training-, 10\% validation-, and 20\% test-data split.
It was decided to use the LightGCN split since this was the method we planned to extend.
\\
After this, we now have a dataset that can be used to compare the different methods with or without category and price.

\subsection{Experimental Settings}
Throughout this subsection the experimental settings and evaluation metrics such as Recall and NDCG are described.

\subsubsection{Parameter settings}
All baseline methods use BPR loss as their loss function and have the embedding size set to 64.
For optimization Adam is used with an initial learning rate of $1e-4$.
BPR is described in Appendix \ref{subsubsec:BPR} and Adam is described in Appendix \ref{subsubsec:Adam}.
The batch size is set to 2048.
For all test in \autoref{subsec:performance-com}, the number of convolutions are set to 3.
The settings for PUP are different as the authors of PUP used other settings.
The batch size for PUP is 1024, and the initial learning rate is $1e-2$.
PUP never specifies in their paper, how many convolutions they used in their experiments, but with the code we received, we came to the conclusion that they only use 1 convolution in PUP.

\subsubsection{Evaluation metrics}
The evaluation metrics used for rating top k recommendation are Recall@K and NDCG@K where K is set to 50 and 100.
Throughout this subsection we describe Parameter settings, NDCG, Recall and Precision.

\subsubsection{NDCG}
Normalized Discounted Cumulative Gain (NDCG) is a measure of ranking quality \cite{NDCG-evaluation}.
To calculate the value of NDCG, it is necessary to first calculate the Discounted Cumulative Gain (DCG).
The equation for DCG is,
\begin{equation}
    DCG = \sum_{i=1}^{n} \frac{relevance_i}{log_2(i+1)},
\end{equation}
where $relevance_i$ is the predicted relevance for item $i$ and where $n$ is the total number of items.
DCG takes the placement into account, so that the array with the most relevant items first in the array will have the largest value \cite{NDCG-evaluation,Handbook}.
An example could be two arrays with the same values. such as $A = [1, 2, 3, 1]$, and $B = [3, 2, 1, 1]$. The results are $DCG(A) = 4.1925$ and $DCG(B) = 5.1925$, where it can be seen that the array with the most relevant items first has the highest value.
NDCG normalizes DCG, so that the recommendations can be compared, as the rating can vary from user to user.
NDCG is calculated as follows,
\begin{equation}
    NDCG = \frac{DCG}{iDCG},
\end{equation}
where $iDCG$ is the DCG with the ideal order \cite{NDCG-evaluation,Handbook}.

\subsubsection{Recall and Precision}
Recall and precision are calculated from the possible outcomes of the recommended items, which can be seen in \autoref{tab:possible-results}.
\begin{table}[]
    \centering
    \begin{tabular}{|l|l|l|}
        \hline
        \rowcolor[HTML]{FFFFFF}
                 & Recommended               & \begin{tabular}[c]{@{}l@{}}Not \\ recommended\end{tabular} \\ \hline
        Used     & \begin{tabular}[c]{@{}l@{}}True-Positive \\ (TP)\end{tabular} & \begin{tabular}[c]{@{}l@{}}False-Negative \\ (FN)\end{tabular} \\ \hline
        Not used & \begin{tabular}[c]{@{}l@{}}False-Positive\\ (FP)\end{tabular} & \begin{tabular}[c]{@{}l@{}}True-negative \\ (TN)\end{tabular} \\ \hline
    \end{tabular}
    \caption{Possible outcomes of a recommendation of an item to a user.}
    \label{tab:possible-results}
\end{table}
Precision describes how precise and accurate the model is from the recommended items, taking $TP$ and $FP$ into account.
Recall describes how accurate the model is related to the items that should be recommended, taking $TP$ and $FN$ into account.
Precision and recall are calculated as follows,
\begin{equation}
    Precision = \frac{\#TP}{\#TP + \# FP}
\end{equation}
\begin{equation}
    Recall = \frac{\#TP}{\#TP + \# FN},
\end{equation}
where \# indicates the total number of $TP$, $FP$, and $FN$.

\subsubsection{Baselines}
The results of the following methods are compared in our experiments.
\begin{itemize}
    \item \textbf{PUP} \cite{Priceaware}: PUP utilizes price and categories to improve the recommendation performance with Graph Convolutional Networks. A more detailed description of PUP can be seen in \autoref{subsec:price-intro}.
    \item \textbf{NGCF} \cite{NGCF_2019}: NGCF utilizes an embedding propagation layer with a Graph Convolutional Network. It was created with the purpose of collaborative filtering. More details can be seen in \autoref{subsec:lightgcn-ngcf}.
    \item \textbf{LightGCN} \cite{lightgcn}: LightGCN was created from NGCF and showed improved performance in Recall and NDCG by removing feature transformations and nonlinear activation function. More details can be seen in \autoref{subsec:lightgcn-ngcf}.
    \item \textbf{GCN} \cite{kipf2017semisupervised}: GCN is a method used for semi-supervised classification on graphs.
    \item \textbf{GC-MC} \cite{berg2017graph}: GC-MC utilizes Graph Convolutional Networks to create the representations for users and items. It only takes the first-order neighbor into account, and therefore only uses one convolution layer.
\end{itemize}

\subsection{Performance comparisons}
As can be seen on \autoref{tab:results-with-many-methods} LightGCN outperforms all the other methods by a significant amount.
With our dataset NGCF performs better than PUP by a small amount, even though Price-Aware recommendation in their paper showcased the opposite \cite{Priceaware}.
Price-Aware recommendation also uses a Yelp dataset for their experiment, however it is not exactly the same dataset.
The results also showcase, that there are a large decrease in performance, when adding prices and categories to the adjacency matrix in LightGCN, which makes it perform worse than any other baseline method.
For NGCF PAS the same changes in input only makes it differentiate negatively by a small amount and actually performs better than PUP.

\begin{table*}[h!]
    \centering
    \begin{tabular}{|l|l|l|l|l|}
        \hline
        \rowcolor[HTML]{FFFFFF}
                       & \multicolumn{4}{l|}{\cellcolor[HTML]{FFFFFF}Yelp Dataset}                                                       \\ \hline
        Method         & Recall@50                                                 & NDCG@50         & Recall@100      & NDCG@100        \\ \hline
        LightGCN       & \textbf{0.2106}                                           & \textbf{0.1063} & \textbf{0.3176} & \textbf{0.1344} \\ \hline
        PUP            & 0.1697                                                    & 0.07802         & 0.2654          & 0.1023          \\ \hline
        GCN            & 0.1558                                                    & 0.07593         & 0.2442          & 0.1001          \\ \hline
        NGCF           & 0.1810                                                    & 0.08817         & 0.2769          & 0.1132          \\ \hline
        GCMC           & 0.1692                                                    & 0.0835          & 0.2497          & 0.1008          \\ \hline
        LGCN PAS (1.0) & 0.1542                                                    & 0.0717          & 0.2199          & 0.086           \\ \hline
        NGCF PAS (1.0) & 0.1749                                                    & 0.0849          & 0.2743          & 0.1111          \\ \hline
    \end{tabular}
    \caption{Results for the experiment with the different methods}
    \label{tab:results-with-many-methods}
\end{table*}

\subsubsection{Hyperparameter experiment}
As described in \autoref{subsec:simple-extension}, we utilize a hyperparameter that is inserted into the adjacency matrix, whenever there is a connection between the item nodes and category nodes, or item nodes and price nodes.
Experiments are done with the hyperparameter $X$ in LGCN PAS and NGCF PAS with the values $0.0$, $0.5$, $1.0$ and $2.0$ to see what effect of this has on the performance.
\begin{table*}[h!]
    \centering
    \begin{tabular}{|l|l|l|l|l|}
        \hline
        \rowcolor[HTML]{FFFFFF}
                       & \multicolumn{4}{l|}{\cellcolor[HTML]{FFFFFF}Yelp Dataset}                                   \\ \hline
        Method         & Recall@50                                                 & NDCG@50 & Recall@100 & NDCG@100 \\ \hline
        LGCN PAS (0.0) & 0.1560                                                    & 0.07674 & 0.2456     & 0.09901  \\ \hline
        LGCN PAS (0.5) & 0.1591                                                    & 0.07825 & 0.2539     & 0.1010   \\ \hline
        LGCN PAS (1.0) & 0.1542                                                    & 0.0717  & 0.2199     & 0.086    \\ \hline
        LGCN PAS (2.0) & 0.1526                                                    & 0.07573 & 0.1581     & 0.06509  \\ \hline
        NGCF PAS (0.0) & 0.1758                                                    & 0.08483 & 0.2737     & 0.1111   \\ \hline
        NGCF PAS (0.5) & 0.1756                                                    & 0.08472 & 0.2744     & 0.1108   \\ \hline
        NGCF PAS (1.0) & 0.1749                                                    & 0.08485 & 0.2743     & 0.1111   \\ \hline
        NGCF PAS (2.0) & 0.1741                                                    & 0.08371 & 0.2762     & 0.1116   \\ \hline
    \end{tabular}
    \caption{Results for the experiment using different input values.}
    \label{tab:hyperparameter-results}
\end{table*}
As can be seen on \autoref{tab:hyperparameter-results} inserting categories and price into the adjacency matrix decreases the performance compared to LightGCN.
The value it has the best performance with is 0.5, followed by 0.0.
Giving prices and categories the same or higher input value as the connection between item and user therefore decreases the results.
There could be multiple reasons for this generally not performing well.
The embeddings created from this input could be used, so that it tries to recommend prices and categories, which is not possible.
Another reason could be because LightGCN does not utilize feature transformation and the nonlinear activation function.
LightGCN performs well, if there are only user and item ids, and the decrease in performance could be because LightGCN is unable to handle the semantic representations in price and category.\\
For NGCF changing the input value of price and categories changes almost nothing.
This is likely because the embeddings for price and categories are never used.
However, we did have an initial idea, that the convolutions with price and category would be reflected onto users and items, which is seen to give a very minimal effect on the outcome.
