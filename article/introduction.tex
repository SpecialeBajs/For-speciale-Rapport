\section{Introduction}
Recommendation systems are widely being used to help users browsing on the internet attenuate information overload\cite{YT_rec,Pint_rec}.
The general idea behind a recommendation system is to predict whether a user will interact with a given item, e.g. purchase, rate, watch, based upon their historical data with similar items.
Collaborative filtering (CF) tackles this problem by assuming that users with similar preferences on some items can be used to recommend each other additional items.
The key functionality behind CF models is to first embed users and items which transforms them to a vectorized presentation.
Secondly, then by using the embedding the historical interactions can be reconstructed.
\\
A very common way of doing CF is by doing Matrix Factorization (MF) as will be explained in \autoref{bsc::MF}.
Though MF models have shown to perform quite well \cite{NGCF_2019} proposes Neural Graph Collaborative Filtering (NGCF).
\\ 
NGCF argues that earlier models do not integrate the user-item interactions especially the bipartite graph structure in their embeddings.
Therefore missing out on collaborative signals which could aid in recommendation.
By utilizing a Graph Convolutional Network (GCN) they are able to achieve great performance compared to other baselines.
\\
GCN's has become the new state-of-the-art for CF but the reason for their effectiveness is not fully understood\cite{lightgcn}.
Through empirical testing \cite{lightgcn} shows that feature transformation and nonlinearactivation, the two most common designs in GCN's, do very little to contribute to more accurate predictions.
They then go on to propose a simple implementation of GCN called LightGCN which achieves even better performance than its more complex GCN counter parts.
\\
Looking at papers such as \cite{Priceaware} it can be seen that propagating the influence of price on users with items as a bridge, aswell as integrating category into the propagation progress increases the performance.
Since \cite{Priceaware} also utilizes a GCN our idea is to improve the performance of the LightGCN framework by utlizing price and category.
