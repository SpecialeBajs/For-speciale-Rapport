\section{Introduction}
Recommendation systems are widely being used to help users browsing on the internet attenuate information overload \cite{YT_rec,Pint_rec}.
The general idea behind a recommendation system is to predict whether a user will interact with a given item, e.g. purchase, rate, watch, based upon their historical data with similar items.
Collaborative filtering (CF) tackles this problem by assuming that users can be used by the recommendation system to recommend items for other users who have similar preferences.
Generally, the most common way to achieve this is to firstly learn latent features to represent a user and an item, this process is also called embedding.
Secondly, the embeddings are used for performing predictions.
\\
An early model for CF is Matrix Factorization (MF), which embeds user and item ID as vectors and uses the dot product to predict whether a user would like a certain item.
Though MF models have shown to perform quite well, \cite{NGCF_2019} proposes Neural Graph Collaborative Filtering (NGCF).
\\
NGCF argues that earlier models do not integrate the user-item interactions especially the bipartite graph structure in their embeddings.
Therefore missing out on collaborative signals which could aid in recommendation.
By utilizing a Graph Convolutional Network (GCN) they are able to achieve great performance compared to other baselines.
\\
GCN's have become the new state-of-the-art method for CF but the reasoning behind their effectiveness is not fully understood \cite{lightgcn}.
Through empirical testing, \cite{lightgcn} shows that feature transformation and nonlinear activation functions, the two most common designs in GCN's, do very little to contribute to the accuracy of the predictions.
They then go on to propose a simple implementation of GCN called LightGCN which achieves even better performance than its more complex GCN counterpart.
\\
Looking at papers such as \cite{Priceaware} it can be seen that propagating the influence of price on users with items as a bridge, as well as integrating categories into the propagation progress increases the performance.
Since \cite{Priceaware} also utilizes a GCN our idea is to improve the performance of the LightGCN framework by utilizing price and category.

We try to extend LightGCN and NGCF by adding price and categories to the embeddings and adjacency matrix.
The code for LightGCN and NGCF and the changes that we made can be seen on Github \footnote{Github link to LightGCN with our extension: https://github.com/SpecialeBajs/LightGCN}.

Since we got the code for PUP sent to us personally through email we chose to not make it publicly available as we do not have the licensing or copyright.
Contact us on email if there are any question surrounding this implementation.
