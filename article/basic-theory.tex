\section{Basic Theory}

\subsection{Graph Convolutional Network}
Graph Convolutional Network (GCN) is a neural network architecture that operates on graphs.
Given a graph $G = (V,E)$ a GCN takes the input of a adjacency
matrix $A$ with size of $N x N$ that represents graph $G$ and a feature matrix $N x F^0$, where $N$ is the total amount of nodes and $F^0$ is the total amount of input features for each node.
A hidden layer in a GCN can be defined as $H^i = f(H^{i-1}, A)$ where $i$ indicates the layer and $H^0$ is the previously mentioned $N x F^0$ feature matrix and $f$ is a propagation function \cite{Deep-Learning-on-Graphs-with-GCN}.
There are many different types of propagation functions.
A simple example could be $f(H^i, A) = \sigma(AH^iW^i) = H^{i+1}$ where $W^i$ is the weight matrix at layer $i$ and $\sigma$ is a non-linear activation function \cite{Deep-Learning-on-Graphs-with-GCN}.
The intuition behind this propagation function is that the future representation of each node is calculated based on its neighbors nodes.
Because of this, each time $i$ is increased, a node's new value is therefore affected not only by its neighbors but its neighbors' neighbors and so on.
An issue with this propagation function could be that the value of each node now is a sum of each of its neighbors, and therefore loses its own value.
This could be solved by replacing $A$ with $\hat{A} = A + I$ where $I$ is the identity matrix.
Doing this the node considers itself a neighbor.
