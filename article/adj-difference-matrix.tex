\subsection{Generated adjacency matrix}
In the following subsection the generated adjacency matrices of LightGCN and price aware recommendation are described.
Both use the adjacency matrix as a representation for a heterogeneous graph, where the value 1 represent a connection between two nodes.

\subsection{LightGCN}
For LightGCN the adjacency matrix is the size of $\textrm{number of users} + \textrm{number of items} \times \textrm{number of users} + \textrm{number of items}$.
Then for every connection between item $i$ and user $u$ 1 is inserted.
It is generated with SciPy that creates a dictionary of keys, where two keys indicate the position in the matrix, and the value is 1 if there is a connection. 

\subsection{Price-aware recommendation}
The adjacency matrix in price-aware recommendation is also implemented as a sparse matrix using SciPy.
This is done by creating two arrays, row and column.
The row and column are both the length of all interactions plus two times the amount of items.
We start by for each user inserting their id as many times to the array as the number of items that they have interacted with.
The id's of the items that the user interacted with are mapped to the corresponding index in the column array.
After all user item interactions have been mapped to the row and column arrays, the item prices and category relationships are mapped to the arrays.
This is done by inserting the index of an item into the row twice.
At the indexes where the item was added to the row, the corresponding category and price related to that item are inserted to the column.
A third array is created with the length of row where ones are inserted.
Using this array, column and row can create a sparse adjacency matrix by mapping the index $i$ in row and column to the array with ones.
Within the adjacency matrix, there will be inserted 1 if there is a connection between item $i$ and user $u$, item $i$ and category $c$, and item $i$ and price $p$.
This results in a adjacency matrix that is of size number of users + number of items + number of categories + number of prices $\times$ number of users + number of items + number of categories + number of prices.
