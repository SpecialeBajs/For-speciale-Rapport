\subsection{Generated adjacency matrix}
In the following subsection the generated adjacency matrices of LightGCN and price aware recommendation are described.

\subsection{LightGCN}
For LightGCN the adjacency matrix is the size of $\textrm{number of users} + \textrm{number of items} \times \textrm{number of users} + \textrm{number of items}$.
Then for every connection between item $i$ and user $u$ 1 is inserted.

\subsection{Price-aware recommendation}
The adjacency matrix in price-aware recommendation is implemented as a sparse matrix using SciPy.
This is generated by creating two arrays row and column.
In the row each user id is added to the array, the number of interactions this user has made.
The column contains the items that the users has interacted with.
The row and column are both the length of all interactions added with two times the number of items.
The row is generated by mapping the user id $i$ into the row the amount of times the user has interacted with an item.
The column is mapped with the corresponding items that the users has interacted with.
At this point the length of row and column, would be equal, as both is the length of the total number of interactions.
When all users and all items has been inserted into the row and column, the prices and categories has to be inserted.
This is done by adding the index of an item into the row twice.
At the indexes of where the item is added to the row, the corresponding category and price related to that item is inserted into the column.
A third array is created with the length of row where ones are inserted.
Using this array, column and row price aware recommendation create a sparse adjacency matrix by mapping the index $i$ in row and column to the array with ones.
Within the adjacency matrix, there will be inserted 1 if there is a connection between item $i$ and user $u$, item $i$ and category $c$, and item $i$ and price $p$.
This results in a adjacency matrix that is of size number of users + number of items + number of categories + number of prices $\times$ number of users + number of items + number of categories + number of prices
