\section{Adjacency matrix comparisons}\label{subsubsec:lightGCN-adj}
In the following subsection the generated adjacency matrices of LightGCN, NGCF and price aware recommendation are described.
All three use the adjacency matrix as a representation for a heterogeneous graph, where the value 1 represent a connection between two nodes.
For LightGCN and NGCF the adjacency matrix is identical.
A more detailed description of the implementation can be seen in Appendix \ref{app:adj-impl}.

\subsection{LightGCN and NGCF}
The adjacency matrix of LightGCN and NGCF is as follows:
\begin{gather}
    A =
    \begin{bmatrix}
        0   & R \\
        R^T & 0
    \end{bmatrix},
\end{gather}
where $R \in \mathbb{R}^{U \times I}$, and $U$ and $I$ denotes the number of users and the number of items summed together.
Each entry in $R_{ui}$ is inserted with 1, if there is a connection between user $u$ and item $i$.
An example can be seen on \autoref{tab:LightGCN-adj}, where $u$ indicates an user and $i$ indicates an item.
As seen it is not possible for a user to have a connection to another user, or an item to connect to another item.
\begin{table}

    \[
        \begin{blockarray}{cccccc}
            & u_1 & u_2 & i_1 & i_2 & i_3 \\
            \begin{block}{c(ccccc)}
                u_1 & 0 & 0 & 1 & 0 & 1   \\
                u_2 & 0 & 0 & 0 & 1 & 1   \\
                i_1 & 1 & 0 & 0 & 0 & 0   \\
                i_2 & 0 & 1 & 0 & 0 & 0   \\
                i_3 & 1 & 1 & 0 & 0 & 0   \\
            \end{block}
        \end{blockarray}
    \]
    \caption{Example of an adjacency matrix in LightGCN}
    \label{tab:LightGCN-adj}
\end{table}

\subsubsection{Price-aware recommendation}
The adjacency matrix of Price-aware recommendation is as follows:
\begin{gather}
    A =
    \begin{bmatrix}
        0   & R \\
        R^T & 0
    \end{bmatrix},
\end{gather}
where $R \in \mathbb{R}^{I \times U+C+P}$, and $I$, $U$, $C$, and $P$ denotes the number of users, number of items, number of categories, and number of prices.
Each entry in $R_{ui}$, $R_{ip}$, $R_{ic}$ is inserted with 1 if there is a connection either between user $u$ and item $i$, item $i$ and price $p$ or item $i$ and category $c$.\\
An example can be seen on \autoref{tab:Price-aware-adj}, where $u$ indicates an user, $i$ indicates an item, c indicates a category, and p indicates a price.
It is not possible to have a connection between an user and a category or price, or for a given type to have a connection with itself.
For example the type of item can not connect to another item.
\begin{table}
    \[
        \begin{blockarray}{cccccccccc}
            & u_1 & u_2 & i_1 & i_2 & i_3 & c_1 & c_2 & p_1 & p_2 \\
            \begin{block}{c(ccccccccc)}
                u_1 & 0 & 0 & 1 & 0 & 1 & 0 & 0 & 0 & 0   \\
                u_2 & 0 & 0 & 0 & 1 & 1 & 0 & 0 & 0 & 0  \\
                i_1 & 1 & 0 & 0 & 0 & 0 & 1 & 0 & 1 & 0  \\
                i_2 & 0 & 1 & 0 & 0 & 0 & 0 & 1 & 0 & 1  \\
                i_3 & 1 & 1 & 0 & 0 & 0 & 1 & 0 & 0 & 1  \\
                c_1 & 0 & 0 & 1 & 0 & 1 & 0 & 0 & 0 & 0  \\
                c_2 & 0 & 0 & 0 & 1 & 0 & 0 & 0 & 0 & 0  \\
                p_1 & 0 & 0 & 1 & 0 & 0 & 0 & 0 & 0 & 0  \\
                p_2 & 0 & 0 & 0 & 1 & 1 & 0 & 0 & 0 & 0  \\
            \end{block}
        \end{blockarray}
    \]
    \caption{Example of an adjacency matrix in Price-Aware recommendation}
    \label{tab:Price-aware-adj}
\end{table}
