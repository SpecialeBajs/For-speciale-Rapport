%%%%%%%%%%%%%%%%%%%%%%%%%%%%%%%%%%%%%%%%%
% Journal Article
% LaTeX Template
% Version 1.4 (15/5/16)
%
% This template has been downloaded from:
% http://www.LaTeXTemplates.com
%
% Original author:
% Frits Wenneker (http://www.howtotex.com) with extensive modifications by
% Vel (vel@LaTeXTemplates.com)
%
% License:
% CC BY-NC-SA 3.0 (http://creativecommons.org/licenses/by-nc-sa/3.0/)
%
%%%%%%%%%%%%%%%%%%%%%%%%%%%%%%%%%%%%%%%%%

\input{preamble.tex}

%----------------------------------------------------------------------------------------
%	TITLE SECTION
%----------------------------------------------------------------------------------------

\setlength{\droptitle}{-4\baselineskip} % Move the title up

\pretitle{\begin{center}\Huge\bfseries} % Article title formatting
\posttitle{\end{center}} % Article title closing formatting
\title{Exploring Recommender Systems with Graph Convolutional Networks} % Article title
\author{%
\textsc{Frederik Valdemar Schrøder, Jens Petur Tróndarson, Mathias Møller Lybech} \\[1ex] % Your name
\normalsize Aalborg University \\ % Your institution
\normalsize \href{mailto:fschra16@student.aau.dk}{fschra16@student.aau.dk} % Your email address
\normalsize \href{mailto:jtrand16@student.aau.dk}{jtrand16@student.aau.dk} % Your email address
\normalsize \href{mailto:mlybec16@student.aau.dk}{mlybec16@student.aau.dk} % Your email address
%\and % Uncomment if 2 authors are required, duplicate these 4 lines if more
%\textsc{Jane Smith}\thanks{Corresponding author} \\[1ex] % Second author's name
%\normalsize University of Utah \\ % Second author's institution
%\normalsize \href{mailto:jane@smith.com}{jane@smith.com} % Second author's email address
}
\date{\today} % Leave empty to omit a date
\renewcommand{\maketitlehookd}{%
\begin{abstract}
\noindent  
Graph Convolutional Network (GCN) is a state-of-the-art method for collaborative filtering (CF).
Throughout this paper we focus on two new methods for CF: LightGCN and Price-aware Recommendation with Graph Convolutional Networks (PUP).
LightGCN is inspired from Neural Graph collaborative Filtering (NGCF), and is a recommender system that only takes the node ID of items and users into account.
PUP also utilize CF, but takes prices and categories into account. 
We take a closer look at the main differences between LightGCN, NGCF and PUP.
After we have found the main differences between these, we extend LightGCN and NGCF by adding prices and categories as nodes in their input graph.
Through our experiments we show that our extension was unsuccesfull in improving the performance of these methods.
In our experiments we run our extension with different hyper parameters to see what effect they have on the results.
We also compare LightGCN to some other state-of-the-art methods and show that it performs significantly better than PUP, NGCF and the other methods.
From the results of these experiments we present some reasons to why our extension performed so poorly.
We end the paper by presenting some possible hypotheses that we can continue to work on in our master thesis.
\end{abstract}
}

%----------------------------------------------------------------------------------------
\lstset{
	basicstyle=\footnotesize,
	numbers=left, 
	numberstyle=\tiny, 
	framexleftmargin=0pt,
	breaklines=true,
	tabsize=2
}

\begin{document}

% Print the title
\maketitle

%----------------------------------------------------------------------------------------
%	ARTICLE CONTENTS
%----------------------------------------------------------------------------------------

\section{Introduction}
Recommendation systems are widely being used to help users browsing on the internet attenuate information overload\cite{YT_rec,Pint_rec}.
The general idea behind a recommendation system is to predict whether a user will interact with a given item, e.g. purchase, rate, watch, based upon their historical data with similar items.
Collaborative filtering (CF) tackles this problem by assuming that users can be used by the recommendation system to recommed items for other users who have similar preferences on items.
Generally, the most common way to achieve this is to firstly learn latent features to represent a user and an item, this process is also called embedding.
Secondly, the embeddings are used for performing predictions.
\\
An early model for CF is Matrix Factorization, which embeds user and item ID as vectors and uses the dot product to predict whether a user would like a certain item.
Though MF models have shown to perform quite well, \cite{NGCF_2019} proposes Neural Graph Collaborative Filtering (NGCF).
\\ 
NGCF argues that earlier models do not integrate the user-item interactions especially the bipartite graph structure in their embeddings.
Therefore missing out on collaborative signals which could aid in recommendation.
By utilizing a Graph Convolutional Network (GCN) they are able to achieve great performance compared to other baselines.
\\
GCN's has become the new state-of-the-art for CF but the reason for their effectiveness is not fully understood\cite{lightgcn}.
Through empirical testing, \cite{lightgcn} shows that feature transformation and nonlinear activation, the two most common designs in GCN's, do very little to contribute to more accurate predictions.
They then go on to propose a simple implementation of GCN called LightGCN which achieves even better performance than its more complex GCN counterparts.
\\
Looking at papers such as \cite{Priceaware} it can be seen that propagating the influence of price on users with items as a bridge, as well as integrating category into the propagation progress increases the performance.
Since \cite{Priceaware} also utilizes a GCN our idea is to improve the performance of the LightGCN framework by utilizing price and category.

We try to extend LightGCN by adding price and categories to the embeddings and adjacency matrix.
The code for LightGCN and the changes that we made can be seen on \footnote{Github link to LightGCN with our extension: https://github.com/SpecialeBajs/LightGCN}.

Since we got the code for \cite{Priceaware} sent to us personally through email we chose to not make it publicly available.
Feel free to contact us on email if there are any question surrounding this.

\section{Method}
This section describes LightGCN, NGCF and price-aware recommendation.
The difference between these recommendation models are described with focus on the adjacency matrix and the embeddings.
An attempt at implementing price aware recommendation in LightGCN and NGCF is also attempted, where the focus is on changing the input parameters.
\subsection{LightGCN and NGCF}\label{subsec:lightgcn-ngcf}
In LightGCN Xiangnan et al. investigate the effect of feature transformation and nonlinear activation functions within collaborative filtering using NGCF \cite{lightgcn}.
NGCF is a framework developed by Wang et al. \cite{NGCF_2019} that utilizes a Graph Neural Network with three components in the framework: (1) an embedding layer that constructs initial user - and item embeddings; (2) embedding propagation layers that capture CF with the two operations of message construction and message aggregation; (3) a prediction layer that concatenates the embeddings at each layer for each user and for each item such that $e_{u}^{(*)} = e_{u}^{(0)}||...||e_{u}^{(l)}$ and $e_{i}^{(*)} = e_{i}^{(0)}||...||e_{i}^{(l)}$ where $u$ indicates the user, $i$ indicates the item, $e$ is the embedding and $l$ is the layer \cite{NGCF_2019}.
Within the second component of message construction the message embedding is implemented as:
\begin{equation}
    m_{u \leftarrow i} = \frac{1}{\sqrt{|\mathcal{N}_u||\mathcal{N}_i|}}(W_1e_i + W_2(e_i \odot e_u)),
    \label{eq:message-construction}
\end{equation}
where $W_1$ and $W_2 \in R^{d' \times d}$ are weight matrices, $d$ is the embedding size and $d'$ is the transformation size.
$\frac{1}{\sqrt{|\mathcal{N}_u||\mathcal{N}_i|}}$ is the graph Laplacian norm, where $\mathcal{N}_u$ and $\mathcal{N}_i$ are the set of neighbors of user $u$ and item $i$.
This is done with the purpose of calculating how much the item contributes to the users preference \cite{NGCF_2019}.
LightGCN criticize the use of the weight matrices $W_1$ and $W_2$ as not being useful for CF as each user-item interaction graph only has the ID as input and it has no semantic value \cite{lightgcn}.
Also within the second component the message aggregation is implemented as:
\begin{equation}
    e_{u}^{(l)} = \mbox{LeakyReLU}(m^{(l)}_{u \leftarrow u} + \sum^{}_{i \in \mathcal{N}} m^{(l)}_{u \leftarrow i}),
\end{equation}
where $e_{u}^{(l)}$ is the embedding of user $u$ at layer $l$.
LeakyReLU is the nonlinear activation function used in NGCF to allow encoding positive and small negative signals.
LightGCN shows that NGCF will perform better if the feature transformation is removed, and that the activation function has small effect when feature transformation is included, but if feature transformation is disabled the activation function will have a negative impact on performance.
LightGCN also shows that NGCF will significantly improve if both the activation function and feature transformation are removed \cite{lightgcn}.

\subsection{Price-aware Recommendation with Graph Convolutional Networks}\label{subsec:price-intro}
Yu Zheng et al. develops an effective method called \textit{Price-aware User Preference-modeling (PUP)} that is used to create recommendations with a focus on the price factor \cite{Priceaware}.
There are two main challenges.
The first is that the preferences of a user on item prices are unknown, and can only be seen implicitly by previous purchases.
The second challenge is that the price of an item largely depends on what category the item is within.
For the first problem they propose a model that creates a relationship between user-to-item and item-to-price using a Graph Convolutional Network, so that they can propagate the influence of price onto the users.
For the second problem they solve this by integrating the categories and items into the propagation process.
PUP represents the data as a heterogeneous undirected graph $G = (V,E)$ where the nodes in $V$ consist of user nodes $u \in U$, item nodes $i \in I$, category nodes $c \in \textbf{c}$, and price nodes $p \in \textbf{p}$.
The edges in $E$ consist of interaction edges $(u, i)$ with $R_{ui} = 1$ if there is an interaction between user $u$ and item $i$ and category edges $(i, \textbf{c}_i)$ and price edges $(i, \textbf{p}_i)$.
Traditional Latent Factor Models such as Matrix Factorization only take a single type of edge $(u, i)$ into account.
This makes them insufficient when multiple types such as $(u, p)$ and $(i, c)$ are introduced \cite{Priceaware}.
Hereby Yu Zheng et al. use a Graph Neural Network to learn the embeddings so that each node has a separate embedding $e' \in \mathbb{R}^d$ where d is the dimensions of the embeddings.
In GCN the nodes propagate their nearest neighbors, which could be user-item, item-price or item-category.
The embeddings are described in further details in \autoref{subsubsec:price}.
The intuition behind the construction of the embeddings is that the node gets aggregated together with its neighbors and in each layer it will go one step further.
So for example a price node embedding will be aggregated together with all item node embeddings connected with this specific price.
By doing this items with the same price levels are expected to be more similar than items not in the same price level.
The intuition about the category is the same as with the price.
The final purchase prediction $s$ between user $u$ and item $i$ is formulated as follows:
\begin{equation}
    \begin{split}
        s & = s_{global} + \alpha s_{category} \\
        s_{global} & = e^T_u e_i + e^T_u e_p + e^T_i e_p \\
        s_{category} & = e^T_u e_c + e^T_u e_p + e^T_c e_p,
    \end{split}
    \label{eq:price-aware-prediction}
\end{equation}
where $e_u$, $e_i$, $e_c$, and $e_p$ are the embeddings for user, item, category and price, respectively.
$\alpha$ is a hyper-parameter used to balance the category branch's influence on the prediction.
This is also what they call their decoder and is a method inspired by factorization machines \cite{Priceaware}.
They have two branches in their decoder which are the global an the category branch.
The global branch only considers the embeddings of user $u$, item $i$ and price $p$.
The category branch is then used to find the relationship between a users preference for price levels within certain categories.
The influence of the category branch is then managed with the hyper parameter $\alpha$.
The embeddings are described in further detail in \autoref{subsubsec:price}.
The results of a Top-K Recommendation Performance test showcase that PUP outperforms other state-of-the-art methods such as NGCF with an average improvement of 3.59 \% to 5.97 \% \cite{Priceaware}.

\section{Adjacency matrix comparisons}
In the following subsection the generated adjacency matrices of LightGCN, NGCF and price aware recommendation are described.
Both use the adjacency matrix as a representation for a heterogeneous graph, where the value 1 represent a connection between two nodes.
A more detailed description of the implementation can be seen in Appendix \ref{app:adj-impl}.

\subsection{LightGCN and NGCF}
The adjacency matrix of LightGCN and NGCF is as follows:
\begin{gather}
    A =
    \begin{bmatrix}
        0   & R \\
        R^T & 0
    \end{bmatrix},
\end{gather}
where $R \in \mathbb{R}^{U \times I}$, and $U$ and $I$ denotes the number of users and the number of items summed together.
Each entry in $R_{ui}$ is inserted with 1, if there is a connection between user $u$ and item $i$.
An example can be seen on \autoref{tab:LightGCN-adj}, where $u$ indicates an user and $i$ indicates an item.
As seen it is not possible for a user to have a connection to another user, or an item to connect to another item.
\begin{table}

    \[
        \begin{blockarray}{cccccc}
            & u_1 & u_2 & i_1 & i_2 & i_3 \\
            \begin{block}{c(ccccc)}
                u_1 & 0 & 0 & 1 & 0 & 1   \\
                u_2 & 0 & 0 & 0 & 1 & 1   \\
                i_1 & 1 & 0 & 0 & 0 & 0   \\
                i_2 & 0 & 1 & 0 & 0 & 0   \\
                i_3 & 1 & 1 & 0 & 0 & 0   \\
            \end{block}
        \end{blockarray}
    \]
    \caption{Example of an adjacency matrix in LightGCN}
    \label{tab:LightGCN-adj}
\end{table}

\subsection{Price-aware recommendation}
The adjacency matrix of Price-aware recommendation is as follows:
\begin{gather}
    A =
    \begin{bmatrix}
        0   & R \\
        R^T & 0
    \end{bmatrix},
\end{gather}
where $R \in \mathbb{R}^{I \times U+C+P}$, and $I$, $U$, $C$, and $P$ denotes the number of users, number of items, number of categories, and number of prices.
Each entry in $R_{ui}$, $R_{ip}$, $R_{ic}$ is inserted with 1 if there is a connection either between user $u$ and item $i$, item $i$ and price $p$ or item $i$ and category $c$.\\
An example can be seen on \autoref{tab:Price-aware-adj}, where $u$ indicates an user, $i$ indicates an item, c indicates a category, and p indicates a price.
It is not possible to have a connection between an user and a category or price, or for a given type to have a connection with itself.
For example the type of item can not connect to another item.
\begin{table}
    \[
        \begin{blockarray}{cccccccccc}
            & u_1 & u_2 & i_1 & i_2 & i_3 & c_1 & c_2 & p_1 & p_2 \\
            \begin{block}{c(ccccccccc)}
                u_1 & 0 & 0 & 1 & 0 & 1 & 0 & 0 & 0 & 0   \\
                u_2 & 0 & 0 & 0 & 1 & 1 & 0 & 0 & 0 & 0  \\
                i_1 & 1 & 0 & 0 & 0 & 0 & 1 & 0 & 1 & 0  \\
                i_2 & 0 & 1 & 0 & 0 & 0 & 0 & 1 & 0 & 1  \\
                i_3 & 1 & 1 & 0 & 0 & 0 & 1 & 0 & 0 & 1  \\
                c_1 & 0 & 0 & 1 & 0 & 1 & 0 & 0 & 0 & 0  \\
                c_2 & 0 & 0 & 0 & 1 & 0 & 0 & 0 & 0 & 0  \\
                p_1 & 0 & 0 & 1 & 0 & 0 & 0 & 0 & 0 & 0  \\
                p_2 & 0 & 0 & 0 & 1 & 1 & 0 & 0 & 0 & 0  \\
            \end{block}
        \end{blockarray}
    \]
    \caption{Example of an adjacency matrix in Price-Aware recommendation}
    \label{tab:Price-aware-adj}
\end{table}
\subsection{Embeddings comparison}
\subsubsection{Price-aware recommendation}\label{subsubsec:price}
The input parameters for construction of the embeddings are the adjacency matrix $A \in \mathbb{R}^{N \times N}$ and identity matrix $I \in \mathbb{R}^{N \times N}$, where $N$ is the number of nodes.
Embeddings from node j to node i are propagated as follows \cite{Priceaware},
\begin{equation}
    t_{ji} = \frac{1}{|\mathcal{N}_i|}e'_j,
\end{equation}
where $\mathcal{N}_i$ is the set of neighbors for node $i$ and $e'_j$ is the embedding node j.
The updating rule is,
\begin{align*}
     & o_u = \sum_{j \in \{i \textrm{ with } R_{ui}=1 \} \cup \{ u\}}^{} t_{ju}                            \\
     & o_i = \sum_{j \in \{u \textrm{ with } R_{ui}=1 \} \cup \{ i, \textbf{c}_i \textbf{p}_i\}}^{} t_{ji} \\
     & o_c = \sum_{j \in \{i \textrm{ with } \textbf{c}_i=c \} \cup \{ c\}}^{} t_{jc}                      \\
     & o_p = \sum_{j \in \{i \textrm{ with } \textbf{p}_i=c \} \cup \{ p\}}^{} t_{jp}                      \\
     & e_f = \textrm{tanh}(o_f), f \in \{u, i, c, p\},
\end{align*}
where $e_u$, $e_i$, $e_c$, and $e_p$ are the embeddings for user $u$, item $i$, category $c$ and price $p$.
These embeddings are used for the prediction as described in \autoref{subsec:price-intro}.

\subsubsection{LightGCN}\label{subsubsec:lightgcn-embedding}
The embeddings for LightGCN are calculated as follows \cite{lightgcn},
\begin{equation}
    E^{(k+1)} = (D^{\frac{1}{2}}AD^{\frac{1}{2}}E^{(k)}),
\end{equation}
where $A$ is the adjacency matrix containing users and items (described in \autoref{subsubsec:lightGCN-adj}) and $D$ is a diagonal matrix, where $D_{ii}$ denotes the sum of the $i-th$ row in the adjacency matrix $A$.
The $0th$ layer embedding $E^{(0)} \in R^{(I + U)\times T}$, where $T$ is the embedding size.
The final embedding is obtained as follows,

\begin{equation}
    e_u = \sum_{k=0}^{K} \alpha_k e_u^{(k)};\;\;\; e_i = \sum_{k=0}^{K} \alpha_k e_i^{(k)},
\end{equation}
where $\alpha$ is set to $1/(K + 1)$.
$\alpha$ is used to normalize the embeddings.

\subsubsection{NGCF}\label{subsubsec:NGCF-embed}
The embeddings for NGCF are calculated as follows \cite{NGCF_2019},
\begin{equation}
    \begin{split}
        E^{(l)} = &LeakyReLU((\lambda + I)E^{(l-1)}W_1^{(l)}+\\
        & \lambda E^{(l-1)}\bigodot E^{(l-1)}W_2^{(l))},
    \end{split}
\end{equation}
where $E^{(l)} \in \mathbb{R}^{I+U \times T}$ are the item and user embedding after $l$ convolutions.
$E^{(0)} \in \mathbb{R}^{I+U \times T}$ is the $0th$ layer embedding, where $T$ is the embedding size.
$I$ is the identity matrix.
$W_1^{(l)}$ and $W_2^{(l)}$ are trainable weight matrices.
$\lambda$ is the Laplacian matrix for the user-item, item-price and item-category graph, and is defined as follows:
\begin{equation}
    \lambda = D^{\frac{1}{2}}AD^{\frac{1}{2}},
\end{equation}
where $A$ and $D$ are the adjacency and diagonal matrix.

\subsubsection{Differences in embeddings}
The difference between NGCF and LightGCN can be seen to be that NGCF utilizes the $LeakyReLU$ activation function as well as using the trainable weight matrices.
Where NGCF and LightGCN are constructed for collaborative filtering without additional side information, the construction of the embeddings in Price-Aware recommendation focus' on creating additional embeddings for both price and category, and using these to make a prediction.
\subsection{LGCN PAS and NGCF PAS}\label{subsec:simple-extension}
\begin{figure}
    \centering
    \includegraphics[scale=0.5]{figures/uipc.png}
    \caption{Illustration of the nodes in the simple extension of LightGCN and NGCF.}
    \label{fig:uipc}
\end{figure}
We extend the implementations of LightGCN and NGCF by changing the input parameters, where we construct the adjacency matrix containing the users, item, category and price graph, which is illustrated in \autoref{fig:uipc}.
This extension for LightGCN is called LGCN PAS and the extension of NGCF is called NGCF PAS.
The intuition behind the idea, is that the graph convolutions will capture the values of categories and price, so even if we do not use the embeddings for price and category, they will still influence users and items.
Let the user-item, item-price and item-category interactions matrix be $R \in \mathbb{R}^{I \times U + C + P}$, where $I$ denotes the number of items, $U, C, P$ denotes the number of users, categories and prices.
Each entry of $R_{ui}$ is 1 if user $u$ has rated item $i$. Otherwise it is 0.
If there is a connection in $R_{ic}$ or $R_{ip}$ this value is a hyperparameter $X$ with a value $x>0$, otherwise it is 0.
The adjacency matrix is obtained as follows:
\begin{gather}
    A =
    \begin{bmatrix}
        0   & R \\
        R^T & 0
    \end{bmatrix}
\end{gather}
The embeddings for users and items are calculated for NGCF as described in \autoref{subsubsec:NGCF-embed} and for LightGCN as described in \autoref{subsubsec:lightgcn-embedding}.

\subsection{Simple price aware extension of NGCF}
To gain an understanding of the effect of feature transformation and nonlinear activation functions a price aware extension has also been added to NGCF.
The embeddings for users and price are calculated as follows:
\begin{equation}
    \begin{split}
        E^{(l)} = &LeakyReLU((\lambda + I)E^{(l-1)}W_1^{(l)}+\\
        & \lambda E^{(l-1)}\bigodot E^{(l-1)}W_2^{(l))},
    \end{split}
\end{equation}
where $E^{(l)} \in \mathbb{R}^{I+U+C+P \times T}$ are the item, user, price, and category embedding after $l$ convolutions.
$E^{(0)} \in \mathbb{R}^{I+U+C+P \times T}$ is the $0th$ layer embedding, where $T$ is the embedding size.
$I$ is the identity matrix.
$W_1^{(l)}$ and $W_2^{(l)}$ are trainable weight matrices.
$\lambda$ is the Laplacian matrix for the user-item, item-price and item-category graph, and is defined as follows:
\begin{equation}
    \lambda = D^{\frac{1}{2}}AD^{\frac{1}{2}},
\end{equation}
where $A$ and $D$ are the adjacency and diagonal matrix, which were constructed as in \autoref{subsec:simple-extension}.



\section{Experiment}
\subsection{Equal data} \label{equal-data}
When measuring the performance of the different methods it is essential that the datasets are the same.
Without them being equal the results of the experiments wont be worth comparing.
Both PUP and LightGCN utilized the yelp dataset according to their papers.
We did not get the dataset from the PUP implementation and were unable to find the original yelp-2018 dataset from the LightGCN implementation were the price and category was still attached.
We therefore decided to utilize the yelp-2020 dataset, in which users rank businesses between 1 to 5 stars.
Inspired by the PUP method we also decided to only use businesses with the restaurant category.
What is left is a set of businesses with one or more subcategories that relate to what type of restaurant they are.
Like PUP we cut off any extra subcategories so that each business now only has one subcategory.
The single subcategory that an item possesses is referred to as its category in the rest of this paper.
\\
PUP has a 60\% training-, 20\% validation- and 20\% test-data split while LightGCN has a 70\% training-, 10\% validation-, and 20\% test-data split.
It was decided to use the LightGCN split since this was the method we planned to extend.
\\
After this, we now have a dataset that can be used to compare the different methods with or without category and price.

\subsection{Experimental Settings}
Throughout this subsection the experimental settings and evaluation metrics such as Recall and NDCG are described.

\subsubsection{Parameter settings}
All baseline methods use BPR loss as their loss function and have the embedding size set to 64.
For optimization Adam is used with an initial learning rate of $1e-4$.
BPR is described in Appendix \ref{subsubsec:BPR} and Adam is described in Appendix \ref{subsubsec:Adam}.
The batch size is set to 2048.
For all test in \autoref{subsec:performance-com}, the number of convolutions are set to 3.
The settings for PUP are different as the authors of PUP used other settings.
The batch size for PUP is 1024, and the initial learning rate is $1e-2$.
PUP never specifies in their paper, how many convolutions they used in their experiments, but with the code we received, we came to the conclusion that they only use 1 convolution in PUP.

\subsubsection{Evaluation metrics}
The evaluation metrics used for rating top k recommendation are Recall@K and NDCG@K where K is set to 50 and 100.
Throughout this subsection we describe Parameter settings, NDCG, Recall and Precision.

\subsubsection{NDCG}
Normalized Discounted Cumulative Gain (NDCG) is a measure of ranking quality \cite{NDCG-evaluation}.
To calculate the value of NDCG, it is necessary to first calculate the Discounted Cumulative Gain (DCG).
The equation for DCG is,
\begin{equation}
    DCG = \sum_{i=1}^{n} \frac{relevance_i}{log_2(i+1)},
\end{equation}
where $relevance_i$ is the predicted relevance for item $i$ and where $n$ is the total number of items.
DCG takes the placement into account, so that the array with the most relevant items first in the array will have the largest value \cite{NDCG-evaluation,Handbook}.
An example could be two arrays with the same values. such as $A = [1, 2, 3, 1]$, and $B = [3, 2, 1, 1]$. The results are $DCG(A) = 4.1925$ and $DCG(B) = 5.1925$, where it can be seen that the array with the most relevant items first has the highest value.
NDCG normalizes DCG, so that the recommendations can be compared, as the rating can vary from user to user.
NDCG is calculated as follows,
\begin{equation}
    NDCG = \frac{DCG}{iDCG},
\end{equation}
where $iDCG$ is the DCG with the ideal order \cite{NDCG-evaluation,Handbook}.

\subsubsection{Recall and Precision}
Recall and precision are calculated from the possible outcomes of the recommended items, which can be seen in \autoref{tab:possible-results}.
\begin{table}[]
    \centering
    \begin{tabular}{|l|l|l|}
        \hline
        \rowcolor[HTML]{FFFFFF}
                 & Recommended               & \begin{tabular}[c]{@{}l@{}}Not \\ recommended\end{tabular} \\ \hline
        Used     & \begin{tabular}[c]{@{}l@{}}True-Positive \\ (TP)\end{tabular} & \begin{tabular}[c]{@{}l@{}}False-Negative \\ (FN)\end{tabular} \\ \hline
        Not used & \begin{tabular}[c]{@{}l@{}}False-Positive\\ (FP)\end{tabular} & \begin{tabular}[c]{@{}l@{}}True-negative \\ (TN)\end{tabular} \\ \hline
    \end{tabular}
    \caption{Possible outcomes of a recommendation of an item to a user.}
    \label{tab:possible-results}
\end{table}
Precision describes how precise and accurate the model is from the recommended items, taking $TP$ and $FP$ into account.
Recall describes how accurate the model is related to the items that should be recommended, taking $TP$ and $FN$ into account.
Precision and recall are calculated as follows,
\begin{equation}
    Precision = \frac{\#TP}{\#TP + \# FP}
\end{equation}
\begin{equation}
    Recall = \frac{\#TP}{\#TP + \# FN},
\end{equation}
where \# indicates the total number of $TP$, $FP$, and $FN$.

\subsubsection{Baselines}
The results of the following methods are compared in our experiments.
\begin{itemize}
    \item \textbf{PUP} \cite{Priceaware}: PUP utilizes price and categories to improve the recommendation performance with Graph Convolutional Networks. A more detailed description of PUP can be seen in \autoref{subsec:price-intro}.
    \item \textbf{NGCF} \cite{NGCF_2019}: NGCF utilizes an embedding propagation layer with a Graph Convolutional Network. It was created with the purpose of collaborative filtering. More details can be seen in \autoref{subsec:lightgcn-ngcf}.
    \item \textbf{LightGCN} \cite{lightgcn}: LightGCN was created from NGCF and showed improved performance in Recall and NDCG by removing feature transformations and nonlinear activation function. More details can be seen in \autoref{subsec:lightgcn-ngcf}.
    \item \textbf{GCN} \cite{kipf2017semisupervised}: GCN is a method used for semi-supervised classification on graphs.
    \item \textbf{GC-MC} \cite{berg2017graph}: GC-MC utilizes Graph Convolutional Networks to create the representations for users and items. It only takes the first-order neighbor into account, and therefore only uses one convolution layer.
\end{itemize}

\subsection{Performance comparisons (RQ1)}\label{subsec:performance-com}
As can be seen on \autoref{tab:results-with-many-methods} LightGCN outperforms all the other methods by a significant amount.
With our dataset NGCF performs better than PUP by a small amount, even though  PUP in their paper showcased the opposite \cite{Priceaware}.
This is due to NGCF performing better in our dataset, compared to PUP's result, while PUP also performing worse.
This could be because of the changes in density in the dataset, which was described in \autoref{equal-data}.
PUP also used a Yelp dataset from 2018 for their experiments, while our Yelp dataset is from 2020 and therefore is a bit larger than theirs.
The results also showcase, that there is a large decrease in performance in LGCN PAS compared to LightGCN, adding price and categories to LightGCN's adjacency matrix makes it perform worse than any of the other baseline methods.
For NGCF PAS it only makes it differentiate negatively by a small amount and actually still performs better than PUP.
\begin{table*}[h!]
    \centering
    \begin{tabular}{|l|l|l|l|l|}
        \hline
        \rowcolor[HTML]{FFFFFF}
                       & \multicolumn{4}{l|}{\cellcolor[HTML]{FFFFFF}Yelp Dataset}                                                       \\ \hline
        Method         & Recall@50                                                 & NDCG@50         & Recall@100      & NDCG@100        \\ \hline
        LightGCN       & \textbf{0.2106}                                           & \textbf{0.1063} & \textbf{0.3176} & \textbf{0.1344} \\ \hline
        PUP            & 0.1697                                                    & 0.07802         & 0.2654          & 0.1023          \\ \hline
        GCN            & 0.1558                                                    & 0.07593         & 0.2442          & 0.1001          \\ \hline
        NGCF           & 0.1810                                                    & 0.08817         & 0.2769          & 0.1132          \\ \hline
        GCMC           & 0.1692                                                    & 0.0835          & 0.2497          & 0.1008          \\ \hline
        LGCN PAS (1.0) & 0.1542                                                    & 0.0717          & 0.2199          & 0.086           \\ \hline
        NGCF PAS (1.0) & 0.1749                                                    & 0.0849          & 0.2743          & 0.1111          \\ \hline
    \end{tabular}
    \caption{Results for the experiment with the different methods}
    \label{tab:results-with-many-methods}
\end{table*}
The recall@50 and NDCG@50 for NGCF, NGCF PAS, LightGCN and LGCN PAS can be seen on \autoref{fig:ndcg-50k-lgcn-ngcf-ngcfpas-pas} and \autoref{fig:recall-50k-lgcn-ngcf-ngcfpas-pas}.
Recall@100 and NCDG@100 for the same methods can be seen on \autoref{fig:ndcg-100k-lgcn-ngcf-ngcfpas-pas} and \autoref{fig:recall-100k-lgcn-ngcf-ngcfpas-pas}.
The figures show that there is not a big difference between the NGCF and NGCF PAS graphs and that NGCF outperforms NGCF PAS by 3.48\% in Recall\@50, 1.04\% in NDCG\@50, 1.01\% in Recall\@100 and 1.02\% in NDCG\@100.
However for LightGCN and LGCN PAS, there is a large difference.
LGCN PAS has a large decrease in performance, and the graph fluctuates a lot, and does not improve when training.
In fact LightGCN performs better by 36.6\% in Recall\@50, 48.3\% in NDCG\@50, 44.4\% in Recall\@100 and 56.3\% in NDCG\@100.
% Overvejer om vi skal have en konklusion på RQ1 eller om det er nok at have det i overskriften. Så ved de jo at det er dette afsnit som besvarer RQ1
\begin{figure}
    \includegraphics[width=\linewidth]{figures/graphs/ndcg-50k-lgcn-ngcf-ngcfpas-pas.png}
    \caption{NDCG$@$50k on LightGCN, NGCF, NGCF PAS and PAS.}
    \label{fig:ndcg-50k-lgcn-ngcf-ngcfpas-pas}
\end{figure}

\begin{figure}
    \includegraphics[width=\linewidth]{figures/graphs/recall-50k-lgcn-ngcf-ngcfpas-pas.png}
    \caption{Recall$@$50k on LightGCN, NGCF, NGCF PAS and PAS}
    \label{fig:recall-50k-lgcn-ngcf-ngcfpas-pas}
\end{figure}

\begin{figure}
    \includegraphics[width=\linewidth]{figures/graphs/ndcg-100k-lgcn-ngcf-ngcfpas-pas.png}
    \caption{NDCG$@$100k on LightGCN, NGCF, NGCF PAS and PAS.}
    \label{fig:ndcg-100k-lgcn-ngcf-ngcfpas-pas}
\end{figure}

\begin{figure}
    \includegraphics[width=\linewidth]{figures/graphs/recall-100k-lgcn-ngcf-ngcfpas-pas.png}
    \caption{Recall$@$100k on LightGCN, NGCF, NGCF PAS and PAS.}
    \label{fig:recall-100k-lgcn-ngcf-ngcfpas-pas}
\end{figure}


\subsection{Hyperparameter experiment (RQ2)}
% Overvejer om vi skal have en konklusion på RQ1 eller om det er nok at have det i overskriften. Så ved de jo at det er dette afsnit som besvarer RQ1
As described in \autoref{subsec:simple-extension}, we utilize a hyperparameter that is inserted into the adjacency matrix, whenever there is a connection between the item nodes and category nodes, or item nodes and price nodes.
Experiments are done with this hyperparameter $x$ in LGCN PAS and NGCF PAS with the values $0.0$, $0.5$, $1.0$ and $2.0$ to see what effect this has on the performance.
\begin{table*}[h!]
    \centering
    \begin{tabular}{|l|l|l|l|l|}
        \hline
        \rowcolor[HTML]{FFFFFF}
                       & \multicolumn{4}{l|}{\cellcolor[HTML]{FFFFFF}Yelp Dataset}                                   \\ \hline
        Method         & Recall@50                                                 & NDCG@50 & Recall@100 & NDCG@100 \\ \hline
        LGCN PAS (0.0) & 0.1560                                                    & 0.07674 & 0.2456     & 0.09901  \\ \hline
        LGCN PAS (0.5) & 0.1591                                                    & 0.07825 & 0.2539     & 0.1010   \\ \hline
        LGCN PAS (1.0) & 0.1542                                                    & 0.0717  & 0.2199     & 0.086    \\ \hline
        LGCN PAS (2.0) & 0.1526                                                    & 0.07573 & 0.1581     & 0.06509  \\ \hline
        NGCF PAS (0.0) & 0.1758                                                    & 0.08483 & 0.2737     & 0.1111   \\ \hline
        NGCF PAS (0.5) & 0.1756                                                    & 0.08472 & 0.2744     & 0.1108   \\ \hline
        NGCF PAS (1.0) & 0.1749                                                    & 0.08485 & 0.2743     & 0.1111   \\ \hline
        NGCF PAS (2.0) & 0.1741                                                    & 0.08371 & 0.2762     & 0.1116   \\ \hline
    \end{tabular}
    \caption{Results for the experiment using different input values.}
    \label{tab:hyperparameter-results}
\end{table*}
As can be seen in \autoref{tab:hyperparameter-results} inserting categories and price into the adjacency matrix decreases the performance in LGCN PAS compared to LightGCN.
The best performing $x$ value is 0.5, followed by 0.0 according to our tests.
Giving prices and categories the same or higher input value as the connection between item and user therefore decreases the performance.
On \autoref{fig:ndcg-pas-weights} and \autoref{fig:recall-pas-weights} it can be seen that the graphs fluctuate with the input values that were used.
There could be multiple reasons for this generally not performing well.
It could be because LightGCN does not utilize feature transformation and does not use a nonlinear activation function.
LightGCN performs well, if there are only user and item ids, and the decrease in performance could be because LightGCN is unable to handle the side information of price and category.
There could also be some contradictions in the data, where LightGCN originally only uses collaborative filtering, and when taking price and category into account, it can contradict each other.\\
For NGCF PAS changing the input value of price and categories changes almost nothing.
This can also be seen on \autoref{fig:ndcg-ngcfpas-weights} and \autoref{fig:recall-ngcfpas-weights}.
This is likely because the embeddings for price and categories are never used.
However, we did have an initial idea, that the convolutions with price and category would be reflected onto users and items, which is seen to give a very minimal effect on the outcome.

\begin{figure}
    \includegraphics[width=\linewidth]{figures/graphs/ndcg-pas-weights.png}
    \caption{NDCG$@$100 on PAS with different input values}
    \label{fig:ndcg-pas-weights}
\end{figure}

\begin{figure}
    \includegraphics[width=\linewidth]{figures/graphs/recall-pas-weights.png}
    \caption{Recall$@$100k on PAS with different input values}
    \label{fig:recall-pas-weights}
\end{figure}

\begin{figure}
    \includegraphics[width=\linewidth]{figures/graphs/ndcg100-ngcfpas-weights.png}
    \caption{NDCG$@$100k on NGCF PAS with different input values}
    \label{fig:ndcg-ngcfpas-weights}
\end{figure}

\begin{figure}
    \includegraphics[width=\linewidth]{figures/graphs/recall-ngcfpas-weights.png}
    \caption{Recall$@$100k on NGCF PAS with different input values}
    \label{fig:recall-ngcfpas-weights}
\end{figure}

\subsection{Experiment with one convolution (RQ3)}\label{sec:experiment-one-convolution}
An experiment was conducted to see what influence changing the number of convolutions to one would have on the performance in NDCG and Recall in the tested methods.
The only method that is not changed is PUP, as it already runs with one convolution.
The results from this experiment can be seen in \autoref{tab:one-convolution}.
For PUP it is the same data as the previous test in \autoref{tab:results-with-many-methods}, as this was already run with only one convolution.
When comparing the results with the previous experiment in \autoref{tab:results-with-many-methods} that used 3 convolutions for all methods, it can be seen that NGCF and LightGCN performs worse when evaluating them with Recall@50 and NDCG@50.
However for NGCF PAS it can be seen that the number of convolutions have little influence on the results.
Its performance decreased with 0.0001 in Recall@50 and improved with 0.00082 in NDCG@50.
LGCN PAS performs worse, which can be seen on \autoref{fig:one-con-ndcg} and \autoref{fig:one-con-recall}.
The results from this experiment show that the benefits of additional convolutions depend on the method.
NGCF PAS does not benefit from the additional convolutions, but for LightGCN and NGCF the additional convolutions improve their performance.
This does not conclude that the more convolutions the better, as too many convolutions can average out the users and items individual values.
\begin{table}[]
    \centering
    \begin{tabular}{|l|l|l|}
        \hline
        \rowcolor[HTML]{FFFFFF}
                       & Recall@50 & NDCG@50 \\ \hline
        LightGCN       & 0.1953    & 0.0957  \\ \hline
        PUP            & 0.1697    & 0.07802 \\ \hline
        NGCF           & 0.1748    & 0.08422 \\ \hline
        LGCN PAS (1.0) & 0.1299    & 0.06454 \\ \hline
        NGCF PAS (1.0) & 0.1748    & 0.08572 \\ \hline
    \end{tabular}
    \caption{Results for experiment with one convolution}
    \label{tab:one-convolution}
\end{table}

\begin{figure}
    \includegraphics[width=\linewidth]{figures/graphs/one-convolution-ndcg.png}
    \caption{NDCG$@$50k with one convolution}
    \label{fig:one-con-ndcg}
\end{figure}

\begin{figure}
    \includegraphics[width=\linewidth]{figures/graphs/one-convolution-recall.png}
    \caption{Recall$@$50k with one convolution}
    \label{fig:one-con-recall}
\end{figure}

\subsection{LGCN PAS performance}
As can be seen from the results in the experiments that we ran, our extension of LightGCN called LGCN PAS performed considerably worse than all of the other methods.
By looking at the graphs \autoref{fig:ndcg-pas-weights} and \autoref{fig:recall-pas-weights} we can see that LGCN PAS does not really improve consistently while training the model.
When the weights are set to 1 the results from the training process seem random and the performance will fluctuate either negatively or positively for every step.
The results of the evaluations fluctuate less when the weights are changed but we still don't see a consistent improvement for each step like we see with the other methods in \autoref{fig:ndcg-50k-lgcn-ngcf-ngcfpas-pas} and \autoref{fig:recall-50k-lgcn-ngcf-ngcfpas-pas}.
There are multiple factors that could be the reasoning behind why LightGCN is unable to handle the additional information added to the adjacency matrix and the embeddings.
The fact that the evaluation of the model while training yields such inconsistent results could indicate that the additional data is causing noise instead of improving the model.
Our initial thought was that this could be caused by the number of convolutions that LightGCN performs on the graph.
The hypothesis was that because there are only 4 price levels and every item is connected to a price level, then the graph would become hyperconnected after a few convolutions.
We ran experiments to test how this affected the results and the conclusion was that LGCN PAS actually performed worse with only one convolution which can be seen in \autoref{sec:experiment-one-convolution}.
Another reason could be that adding additional information to the embeddings creates noise that disturbs the collaborative signals in the model.
In PUP they use a method inspired by factorization machines to factorize the embeddings and control how much each type of embedding should affect the recommendation.
Our approach is quite naïve compared to how PUP handles the embeddings which is most likely the reason behind why our method did not perform very well.

\section{Related work}
In this section we take a look at existing work that relates to our planned extension of the Light GCN approach.

\subsection{Collaborative Filtering}
Collaborative Filtering (CF) is one of the most prevalent techniques in current recommendation systems\cite{YT_rec,NGCF_2019,Pint_rec}.




\subsection{Graph-Based Recommendation}





\section{Future work}
For our master thesis, we will continue to explore recommendation systems and continue in the path of LightGCN and Price-aware recommendation.
There are several topics related to these implementations to be investigated to gain a better understanding of how recommender systems can be improved.
The hypothesis for the master thesis are:
\begin{itemize}
    \item Adding price and category embeddings to LightGCN will improve the performance of the simple price-aware extension.
    \item If adding price and category embeddings improves the performances, then it can get generalized to include other features than price and category.
    \item Price-aware recommendation can be generalized so other features can get integrated to improve the recommendation performance.
    \item Extending categories so one item can have multiple categories will better capture cross-category price preferences.
\end{itemize}

\subsection{Adding price-aware embeddings to LightGCN}
An extension of the embeddings in LightGCN could be implemented by combining the embeddings from the different methods.
Inspired by utilizing \autoref{eq:price-aware-prediction}, we could use the embeddings for price and category.
An alternative could be trying to combine the original LightGCN embedding without price and category for users and items, as these capture the collaborative filtering quite well and then adding the price and category embeddings.
A possible outcome of this can however be that there still are contradictions between the collaborative signals, and price and category signals.
It can also be that the missing weight matrices and activation function in LightGCN that is used in Price-Aware recommendation and NGCF does not make it suited for handling side information.

\subsection{Generalizing feature inputs for LightGCN or Price-Aware recommendation}
The intuition behind Price-Aware recommendations can easily be put to use for other datasets with other types of data.
Doing it for the adapted LightGCN requires that adding price and category embeddings gives a reasonable result before it makes sense to generalize it.
If extending LightGCN is not feasible, then NGCF could be extended, as this has shown interesting results.
An example could be if we try to recommend movies to users, then instead of price and category, these could be replaced with actors or genre, or other types of data.
However, other types of data, such as movies with actors are not necessarily as related to price and category.
For prices and categories, a user can have different preferences for the price range in different categories.
The same can apply for actors in movies, where users prefer actors in a specific genre, but these are likely less related than prices and categories.
Therefore we would need to experiment with different graphs, where users could be connected to items, genres, or actors, and where the item is a bridge between genres, actors, and users.
An example of a graph that would require experiments could be if the user-item, item-actor, and user-actor graph would perform better than a user-item and item-actor graph.

\subsection{Extending categories}
To prove their concept PUP decided to only retain one category for each item even though the dataset used has multiple categories for each item.
Having multiple categories might help to better capture cross-category price preferences.
It might also increase noise by having too much information to accurately predict items.
This could be alleviated by having categories in a ranked list and having weights that try to reduce the influence of too many categories on one item.

\section{Conclusion}
In this research paper we investigated some state-of-the-art GCN recommendation methods for collaborative filtering.
We focused on LightGCN, NGCF, and PUP, where we described differences in the construction of the adjacency matrices and the differences in the construction of the embeddings.
\\
One of our early hypotheses was that by simply changing the input we could achieve a higher precision.
Inspired by the PUP method we extended the LightGCN and NGCF methods with price and category as additional side information.
The methods were changed to test the effect of price and category as input values.
This showed that only adding this data to the adjacency matrices was not sufficient to increase the performance of these methods and actually made them perform worse.
\\
We conducted experiments which showed that LightGCN outperforms other methods such as PUP and NGCF.
However, the extensions LightGCN- and NGCF-PAS showed that NGCF is better suited for handling additional side information, where LightGCN currently does not handle additional side information well.
\\
Our plans for the master thesis are to continue working with LightGCN, NGCF, and PUP and to explore the hypotheses we developed.

\begin{appendices}
	\input{appendix/basic-theory.tex}
	\section{Python libraries for implementing neural networks}
During this project, we examined two implementations of recommendation systems which were LightGCN and Price aware recommendation with graph convolutional networks(PUP). 
The implementations are different from each other because they use different libraries when implementing the neural networks.
LightGCN uses TensorFlow and PUP uses PyTorch.

\subsection{PyTorch}
PyTorch has an object-oriented approach to creating neural networks.
A lot of the implementation is already done for you and you for the most part just need to set up the boilerplate code.
You start by creating a class that inherits from a class in PyTorch called \textit{nn.module}.
In the constructor of the class, you initialize all of the parameters in the network as tensors from PyTorch.
Finally you have to implement the forward function of the network which is used to process the input through the network.
The backward function that computes the gradients is already implemented.
After you have called the network with a given input you then can call the \textit{backward} function on the result to backpropagate the error.
PyTorch also has implementations of different update rules which can be used through their package called \textit{optim} which can be used to update the weights of the neural network after you have backpropagated the error.

\begin{lstlisting}[language=Python]
	import torch
	import torch.nn.Parameter import Parameter

	class NeuralNet(torch.nn.module):
		def __init__(self):
			super(NeuralNet, self).__init__()

			#initializes a tensor of the size 1000 x 64 as a Parameter in the neural network
			self.weights = Parameter(torch.FloatTensor(1000, 64))
		
		def forward(self, input):
			return torch.spmm(input, weights)

\end{lstlisting}

\begin{lstlisting}[language=Python]
	import torch
	import torch.optim

	optimizer = optim.SGD(net.parameters(), lr=0.01) #neural net optimizer using SGD update rules
	neural_net = NeuralNet()
	
	...

	def train_one_loop(input):
		#set the gradient buffers to zero. Otherwise they accumulate after every loop
		optimizer.zero_grad() 

		output = net(input)

		loss = our_loss_function(output)

		#backpropogates the error
		loss.backward() 

		#updates the weights in the neural net
		opitimizer.step() 

\end{lstlisting}

\subsection{TensorFlow}
Implementing a neural network in TensorFlow is very different from implementing a neural network in PyTorch.
TensorFlow does not use an object-oriented approach so you have to define everything manually.
You start by defining every variable that will be present in the neural network.
This might be the size of the input layer, hidden layers and output layer.
If you don't have a value to insert for them you still need to define placeholders so that TensorFlow has an idea of the shape and type of them.
After defining all of the variables you need to define the cost of the neural networks and the optimizer which can be the back propagating algorithm.
Just like in PyTorch TensorFlow has some optimizers that are already implemented for you. 
After you have defined all of the variables in your neural network you can initialize all of them by calling \textit{$initialize_all_variables$}.
You can then start a training session by instantiating the Session class in TensorFlow.
To start training you the call run function on the session object with the initialized variables.
From here you can start training by calling the run function again with the input data, the cost and the optimizer.
Every time we pass some input through the neural network with a cost and an optimizer it will back propagate and update the weights in the neural network.

\begin{lstlisting}[language=Python]
	import tensorflow as tf

	input_size = 1032
	hidden_size = 500
	output_size = 10
	epochs = 5
	batch_size = 128
	learning_rate = 0.01
	seed = 128

	#define placeholders
	input = tf.placeholder(tf.float32, [None, input_size])
	output = tf.placeholder(tf.float32, [None, output_size])

	#define weights and biases of neural network
	weights = {
		'hidden': tf.Variable(tf.random_normal([input_size, hidden_size], seed=seed)),
		'output': tf.Variable(tf.random_normal([hidden_size, output_size], seed=seed))
	}

	biases = {
		'hidden': tf.Variable(tf.random_normal([hidden_size], seed=seed)),
    	'output': tf.Variable(tf.random_normal([output_size], seed=seed))
	}

	#create neural network computational graph
	hidden_layer = tf.add(tf.matmul(x, weights['hidden']), biases['hidden'])
	hidden_layer = tf.nn.relu(hidden_layer)

	output_layer = tf.matmul(hidden_layer, weights['output']) + biases['output']

	#define cost
	cost = tf.reduce_mean(tf.nn.softmax_cross_entropy_with_logits(output_layer, y))

	#define optimizer
	optimizer = tf.train.AdamOptimizer(learning_rate=learning_rate).minimize(cost)

	#after everything is set up we need to initialize all the variables
	init = tf.initialize_all_variables()

	#create a session and start learning
	with tf.Session() as session:

		#create the initialized variables in the current session
		session.run(init)

		#start training
		for epoch in range(epoch)
			avg_cost = 0

			#training data is divided into batches of size batch_size
			#here we get the number of batches we will go through
			num_of_batches = int(size_of_training_data / batch_size)
			
			for i in range(num_of_batches)
				#some function we have implemented that gets us one batch of training data
				batch = our_get_batch_function(i)

				#run the batch through the neural network
				_, c = session.run([optimizer, cost], feed_dict = {x: batch})

				avg_cost += c / total_batch
\end{lstlisting}

\end{appendices}

\printbibliography[heading=bibintoc]
\label{bib:mybiblio}
%----------------------------------------------------------------------------------------
%	REFERENCE LIST
%----------------------------------------------------------------------------------------


%\begin{thebibliography}{99} % Bibliography - this is intentionally simple in this template
%
%  \bibitem[Figueredo and Wolf, 2009]{Figueredo:2009dg}
%  Figueredo, A.~J. and Wolf, P. S.~A. (2009).
%  \newblock Assortative pairing and life history strategy - a cross-cultural
%  study.
%  \newblock {\em Human Nature}, 20:317--330.

%\end{thebibliography}

%----------------------------------------------------------------------------------------

\end{document}
